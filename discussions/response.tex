\documentclass[preprint]{imsart}
%\bibliographystyle{asa}
\usepackage{fullpage}

\usepackage[utf8]{inputenc}
\usepackage[pdftex]{graphicx}
\DeclareGraphicsExtensions{.png,.pdf}
\graphicspath{{3-classifly/}{3-hclust/}{3-manova/}{3-tools/}{4-meifly/}{5-projection-pursuit/}{5-som/}{6-nnet/}}

\usepackage{hyperref}
\usepackage{color}
\definecolor{slateblue}{rgb}{0.07,0.07,0.488}
\hypersetup{colorlinks=true,linkcolor=slateblue,anchorcolor=slateblue,citecolor=slateblue,filecolor=slateblue,urlcolor=slateblue,bookmarksnumbered=true,pdfview=FitB}

\usepackage[usenames,dvipsnames]{xcolor}
\newcommand{\hh}[1]{{\color{ForestGreen} #1}}


\usepackage[small]{caption}
\usepackage{url}
\usepackage[round,sort&compress,sectionbib]{natbib}
\usepackage{amsmath}

\startlocaldefs
\DeclareMathOperator{\Normal}{Normal}
\DeclareMathOperator{\logit}{logit}
\endlocaldefs

\begin{document}

\begin{frontmatter}
\title{Authors' Response to Discussants}
\runtitle{Visualizing statistical models}
\begin{aug}
\author{\fnms{Hadley} \snm{Wickham}\corref{}\ead[label=e1]{hadley@rice.edu}},
\author{\fnms{Dianne} \snm{Cook}\ead[label=e2]{dcook@iastate.edu}}
\and
\author{\fnms{Heike} \snm{Hofmann}\ead[label=e3]{hofmann@iastate.edu}}

\affiliation{Rice University}
\address{Department of Statistics MS-138\\6100 Main St\\Houston TX 77081\\ \printead{e1}}

\affiliation{Iowa State University}
\address{Department of Statistics\\2415 Snedecor Hall\\Ames IA 50011-1210\\ \printead{e2}}

\affiliation{Iowa State University}
\address{Department of Statistics\\2413 Snedecor Hall\\Ames IA 50011-1210\\ \printead{e3}}

\end{aug}
\end{frontmatter}

We very much appreciate the insights and additional material provided by the discussants. In the world of big data, visualization becomes so  important on many fronts - communication of information to a broad audience, diagnosis of analytical methods, exploration - but it becomes increasingly nebulous with large volumes of data. Using analytical procedures to process the data can make a problem initially tractable. 

Allen et al (2015)  provides an example of the principle "visualizing the process of model fitting" outlined in our paper, successfully applied to big data. In the process of visualizing distributed optimizations, they realized that the same results could be obtained by a single optimization using a range of parameters, which was further validated by theoretical work. It was the visualization that led them to this discovery. They also provide areas in machine learning with big data where "visualizing members of the collection" could be applied to yield better models. The third principle of "visualizing the model in the data space" is where they see less scope for use with big data. They outline several key issues - overplotting of huge numbers of points, distributed data, mixed data types, high-dimension with few samples - which provide challenges for applying the principle. And we agree. We have given some examples for data containing only quantitative variables. So we issue these as challenges to new statistical graphics and data visualization researchers, to advance our methodology so that glimpses of the model in the data space can be made. What is important is to build a synergy of model fitting, and visualization, and where possible show components of the model in some space of the data. The methods described in the various articles written by Simon Urbanek (referenced in main article)  illustrate some starting points. 

Leek et al (2015) represent the main stream view of statistics. We agree that the methods described in the paper are very useful for teaching statistical concepts and machine learning algorithms, too. You can find videos and lecture notes at \url{http://streaming.stat.iastate.edu/~dicook/EDA.and.datamining/} to show how to tease apart the results of a random forest model, among many other examples. There are also videos available at \url{https://vimeo.com/user14048736} that have been created to help students understand various multivariate analysis models and tests. We disagree the commentary in terms of the scope of visualization in the practice of statistics. We believe that visualization needs to be better integrated in all areas of statistical practice. Conceptually, we believe that data analysis and modeling is a cyclical process, as shown in Figure \ref{hadley-diagram} where visualization feeds back mode improvement information back  into the model.  Leek et al have missed the point on the all models perspective. Plotting a single scatterplot of living area and number of bedrooms hints at potential collinearity problems for model fitting, but the correlation does not look so strong, and looking at these two variables only negates the interplay between the multiple predictors. Its funny that given their arguments for people having a difficult time perceiving correlation from a plot, that they would advocate the single scatterplot is better than the all models view. The scatterplot itself doesn't tell us how to fix the model. The all models view tells us that the "best" models as measured by $R^2$ contain living area, and bedrooms, albeit having a practically difficult yo explain negative coefficient for bedrooms. It says that bedrooms are important. Leek et al neglected to describe the next steps for the modeling, which are to partially regress bedrooms on living area and only consider the residuals from this fit as the predictor. Fitting the all models would help new students better understand the complications arising from multiple predictors. 

\centerline{\em Hadley, Can you put your usual diagram of the data analysis cycle here?}

Buja's post-model inference, ggobi intro, 

Statistical significance is cheap today

\bibliography{../references}

\end{document}